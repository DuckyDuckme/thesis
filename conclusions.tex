We have shown few cryptographic lattice based schemes and how we can use them to create a fully homomorphic encryption. There were few things we left out to either make this exposition simpler or due to the time constraints of this bachelor thesis. These include, for example, taking the secret $s$ from the dual $\Rd_q$ rather than the ring $R_q$ itself. Since we have taken the underlying ring to be cyclotomic of a power of 2, ther are in fact equivalent up to scaling as motivated at the beginning of the section on ring-LWE. Another simplification was taking the error from the same structure as the $a$ and $s$ itself rather than some smaller space like the $[0,1)$ in the case of LWE and $\mathbb{T} := K \otimes \R$ in the ring-LWE setting. They are also ``equivalent'' with some minor loss of generality since the former do not depend on $q$ unlike our choices of domain for the error distribution.

There is also the question of choices of our tools like for example the metric. We show that this specific choice of parameters makes our Gaussian distribution close to uniform. But this is only the case for our choice of metric.

Another thing that is missing is the ``squashing'' introduced by Gentry. This simply involves including a Sparse Subset Sum (\prob{SSSP}) of the secret key in the public key to ease the decryption procedure and enable bootstrapping. It intoduces the assumption of \prob{SSSP} to his construction and can be regarded as an archetype of his construction since the newer schemes are simply more efficient (including their decryption algorithm/circuit). These include for example the aforementioned \cite{fhe_rlwe}.


%In general, the area is relatively young compared to the number-theoretic approach. It is definitely equivalently rich and beautiful.  There are many things we still need to understand like for example for the cryptography so far. There are many However, as long as
%a quote from \cite{intro_cryp}: "In real world scenarios, cryptosystems based on N P-hard or N P-complete problems tend to rely on a particular subclass of problems, either to achieve efficiency or to allow the creation of a trapdoor. When this is done, there is always the possibility that some special property of the chosen subclass of problems makes them easier to solve than the general case"
