a quote from \cite{intro_cryp}: "In real world scenarios, cryptosystems based on N P-hard or N P-complete problems tend to rely on a particular subclass of problems, either to achieve efficiency or to allow the creation of a trapdoor. When this is done, there is always the possibility that some special property of the chosen subclass of problems makes them easier to solve than the general case"
\begin{remark}
    We are still faced with a problem that is inherent to all of modern-day cryptography. That is, we are assuming the hardness of the problem based on our inability to efficiently solve it. As correctly trivialized by Daniel J. Bernstein \cite{bernstein}: ``nobody has figured out an attack so we conjecture that no attack exists''. It might so happen that tomorrow someone finds an efficient (polynomial time) algorithm to find the shortest vector in a given lattice and our secrets are compromised. This is exactly what happened in the case of RSA cryptosystem when Shor found such efficient algorithm for integer factorization. There is not much we can do about it at least with our current approach to cryptography which is based on very precise complex-theoretic assumptions. This is because complexity theory does not provide any tools to prove that an efficient algorithm does not exist for any given problem.
\end{remark}
