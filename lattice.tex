Gentry's work was a true breakthrough. It not only presented the first, fully homomorphic encryption scheme, but also gave researchers a very powerful tool the \textit{bootstrapping} is. From now on, all we need to construct another FHE scheme, is some suitable (one requirement would be to use a scheme based on ring rather than a group) SHE method, apply appropriate "squashing" to obtain the bootstrapping and we are done. In the following years this is exactly what happened in the academia and the industry. This section will mostly serve as a survey of the main developments towards more efficient fully homomorphic encryption using lattices and their security based on computational hardness of the underlying problems.

\subsection{The GGH public key cryptosystem}
We will start this section with a somewhat simpler cryptosystem that was developed by Goldreich, Goldwasser and Halevi in late-1990s \cite{ggh} called the GGH cryptosystem. This scheme, rather than using ideal lattices (i.e. lattices that are also ideals in the ring of integers), relies on general properties of lattices. Namely, the hardness of the SVP and CVP (see section \ref{hardness}).

\subsubsection*{Idea behind the scheme}
The basic GGH cryptosystem, as mentioned before, is based on problem of finding the closest vector in the lattice $\mathcal{L}$ to a given point in the ambient space $\R^n$. We are given two bases, call them $\Bg$ and $\Bb$. The $\Bb$ will be our public key and $\Bg$ the secret key. Our secret message $\bm{m}$ is represented as a binary vector which we will use to form a linear combination $\bm{s} = \sum m_i \bm{v}_i^{bad}$ of the vectors in $\Bb$. We now add some small and random\footnote{some small note about the "randomness" of this e} error $\bm{e}$ to obtain the ciphertext $\bm{c} = \bm{s} + \bm{e} = \sum m_i \bm{v}_i^{bad} + \bm{e} \in \R^n$ - some point that is not in the lattice, but rather, very close to a point in it.
To decrypt, we can use our good basis $\Bg$ to represent $\bm{c}$ and Babai's algorithm\footnote{Simply stated, if the vectors of the basis are sufficiently orthogonal to one another, then this algorithm solves CVP. However, if the Hadamard ratio is less than <fill in>, the algorithm fails to find the closest vector \cite{babai}} to find $\bm{v}$ and represent it in terms of the basis $\Bg$ \krzys{i think i found a mistake in the book by Hoffstein et al coz it says there "bad" instead of "good"} to recover $\bm{m}$. On the other hand, any eavesdropping adversary that is trying to learn our secret, is left with some bad basis that will be of no help in solving the CVP.

\subsubsection*{GGH construction - concretely}
$\alg{KeyGen}$:
\begin{itemize}
    \item Pick a basis $\B = \{\bm{v}_1, \bm{v}_2, \dots, \bm{v}_n\} \subset \Z^n$ such that they are reasonably orthogonal to one another - i.e. with small Hadamard ratio. We will associatie the vectors $\bm{v}_1, \bm{v}_2, \dots, \bm{v}_n$ as the $n$-by-$n$ matrix $V$ and let $\mathcal{L}$ be the lattice generated by these vectors. This is our good basis $\Bg$ - the \textbf{private key}.
    \item Pick an $n$-by-$n$ matrix $U$ with integer coefficients and determinant $\pm 1$ and compute $W = UV$. The row vectors $\bm{w}_1, \bm{w}_2, \dots, \bm{w}_n$ of $W$ are the bad basis $\Bb$ of $\mathcal{L}$ - the \textbf{public key}.
\end{itemize}
\begin{remark}
    The matrix $U$ is chosen with $\det (U) = \pm 1$ so that the $\det \mathcal{L} = \text{Vol} (\mathcal{F}(\bm{v}_1, \dots, \bm{v}_n)) = | \det (W) | = | \det(UV) | = | \det(U) | \cdot | \det(V) | = | \det(V) |$. This ensures that we do not alter the $\text{Vol} (\mathcal{F})$ (see definition \ref{fundamental}) and thus we are left with the original lattice $\mathcal{L}$.
\end{remark}

Toward basing fully homomorphic encryption on worst-case hardness
lll

\subsection{Fully Homomorphic Encryption Using Ideal Lattices}
explain here how we can construct a really nice homomorphic encryption scheme using ideal lattices \cite{gentry}. present the 
\subsection{On Ideal Lattices and Learning With Errors Over Rings}
this is somewhat too difficult for me i think so ill just present main findings without proofs and details \cite{regev}, \cite{ring-lwe} \\
First explain what lattices are. \\
How do lattices relate to LWE? The secret key is associated with a random vector. \\
then show how ring-lwe satisfies both of our requirements \cite{ring-lwe}, namely, the believed hardness for quantum computers (SVP or approximate SVP) and FHE. Show also the problem with ring-LWE because the lattices that are used there are ideal lattices which obviously possess more structure than "normal" lattices.
