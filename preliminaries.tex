\subsection{Notation}
Most of the notation is ``standard'' in the field of number theory and cryptography. Nonetheless, to avoid any misunderstandings and simplify some statements, we will present the notation used throughout the text. Anything that is not mentioned here shall be defined ``on the go'' with definitions and such. \\

\noindent Any scalars $a, t, \beta, \dots$ are represented by non-bold, latin or greek letters.\\
Bold symbols $\bm{v}, \bm{x}, \bm{s}, \dots$ will denote vectors. \\
$\alg{Algorithms}$ will be represented by $\alg{lmodern font}$ names such as $\alg{Encrypt}$ or $\alg{Evaluate}$.\\
Traditionally, the symbols $\Z, \Q, \R, \C$ shall represent the sets of integer, rational, real and complex numbers, respectivelly.\\
Additionally, for any real $m \geq 0$, $\flo{m}$ shall denote greatest integer not exceeding $m$, $\round{m} = \flo{m + \frac{1}{2}}$ and $[m]$ is the set $\{1, 2, \dots, \flo{m}\}$.\\
When the situation demands us to make use of many variables and/or indicies, we will be using Eisenstein notation which is defined as \krzys{fill that in}

\subsection{Lattices}
\subsubsection*{Basic Definitions}
We define a \textit{lattice} as a discrete additive subgroup of $\R^n$. Once we fix a basis $\B = (\bm{b}_1, \dots ,\bm{b}_n) \in \R^n$ we can then describe the lattice as
$$ \Lambda = \mathcal{L}(\bm{B}) = \Bigl\{ \sum_i z_i \bm{b}_i : z_i \in \Z \Bigl\}.$$

There are many bases for a lattice (actually, for $n \geq 2$, there are infinitely many as can be proven using a diagonalization argument), some ``better'' than others. This will be the foundation for some of the cryptosystems later like the GGH.

\begin{example}
    The simplest example of a lattice is the $\Z^n$ itself. Taking the standard basis $\B_1 = (\bm{e}_1, \dots, \bm{e}_n)$ we obtain
$$\mathcal{L}(\bm{B}_1) = \Bigl\{ \sum_i z_i \bm{e}_i : z_i \in \Z \Bigl\} = \Z^n.$$
\end{example}
More generally, $\Lambda$ is a lattice of rank $m$ in $\R^n$ if it is a rank $m$ free abelian group. Recall that we call a group \textit{free abelian group} of rank $m$ if it can be written as $\Lambda = \Z\beta_1 \oplus \cdots \oplus\Z\beta_m$ with $\beta_1, \dots, \beta_m$ linearly independent over $\R$ where $\oplus$ represents the direct sum. In this paper we will only consider lattices of full rank $n$. 

\begin{remark}
    We can also view the vectors $\bm{b}_i$ as the columns of the matrix $\B \in \R^n \cross \R^n$ in which case, our definition becomes:
    $$\Lambda = \mathcal{L}(B) = \{\bm{Bz} :  \bm{z} \in \Z^n \}.$$
\end{remark}

Reciprocally, any matrix $\bm{B} \in GL_n(\R)$ spans a lattice: the set of all integer linear combinations of its rows.

\begin{example}
\begin{enumerate}
    \item $\mathcal{L} = \begin{pmatrix}
        1 & 0\\
        0 & 1
	\end{pmatrix}$ in which case $\bm{b}_1 = \big(\begin{smallmatrix}
          1\\
          0
	\end{smallmatrix}\big)$ and $\bm{b}_2 = \big(\begin{smallmatrix}
          0\\
          1
        \end{smallmatrix}\big)$
    \item $\mathcal{L} = \{(z_1,z_2) : z_1 + z_2 \text{ is even}\}$
    \item $\mathcal{L} = \begin{pmatrix}
        13 & 21\\
        21 & 34
        \end{pmatrix}$
\end{enumerate}
\end{example}

As noted before, the basis of a lattice is not unique. There is one that is particularly interesting to us, namely, the \textit{Hermite Normal Form} (HNF). A basis $\B$ is in HNF if it is upper triangular (or lower triangular - does not matter as long as one is consistent), all elements on the diagonal are strictly positive and any other element $\bm{b}_{i,j}$ satisfies $0 \leq \bm{b}_{i,j} < \bm{b}_{i,i}$.

\subsubsection*{Fundamental Domain}
\begin{definition}[Fundamental Domain] \label{fundamental}
    Let $\mathcal{L}$ be a lattice of dimension $n$ and let $(\bm{b}_1, \dots, \bm{b}_n)$ be a basis for $\mathcal{L}$. The \textit{fundamental domain} (or \textit{fundamental parallelepiped}) for $\mathcal{L}$ corresponding to this basis is the set
    $$ \mathcal{F}(\bm{b}_1, \dots, \bm{b}_n) = \{t_1\bm{b}_1 + \cdots + t_n\bm{b}_n : 0 \leq t_i < 1 \}.$$
\end{definition}

We define the \textit{volume} of $\mathcal{F}(\bm{B})$ as the volume of the corresponding parallelepiped in $\R^n$. The \textit{volume} - closely connected to the determinant - plays a very important role in our study which will become evident in later chapters. One of the advantages, of defining the fundamental domain, is that we can formalize the notion of area (or the determinant) of any given lattice. Recall that a lattice is just a countable collection of points and therefore has no volume by itself. This, however, is resolved by introducing the following.

\begin{definition}
    Let $\mathcal{L}$ be a lattice of dimension $n$ and let $\mathcal{F}(\bm{B})$ be a fundamental domain for $\mathcal{L}$ over some basis $\bm{B}$. We define the \textit{determinant} of that lattice as
    $$ \det (\mathcal{L}) = \VF{\bm{B}} = |\det (\bm{B}) |$$
\end{definition}

The next two propositions are \textit{de facto} foundation for lattice based cryptography. The first one states that the $\det (\mathcal{L})$ does not depend on the choice of the basis for that lattice. The second, that our whole ambient space $\R^n$ can be described using only vectors from the lattice and the fundamental domain. We will only give an outline of the proofs for the sake of keeping this section compact. Full proofs, however, can be found in \cite{book}, chapter 6.4.

\begin{proposition}
    The $\det (\mathcal{L})$ of an $n$-dimensional lattice is invariant under the choice of the basis.
\end{proposition}

\begin{proof}[Outline of the proof]
    Let $\bm{B}_1, \bm{B}_2$ be two bases for a lattice $\mathcal{L}$. The crucial part of the proof is to note that any two bases are related by some unimodular matrix $U$ (i.e. a matrix with the determinant of $\pm 1$) s.t. $\bm{B}_1 = U \bm{B}_2$. It now easily follows to compute $| \det (\bm{B}_1) | = \det (\mathcal{L}) = | \det (U \cdot \bm{B}_2) | = | \det(U) | \cdot | \det(\bm{B}_2) | = | \det(\bm{B}_2)|$ 
\end{proof}

From now on we will write $\mathcal{F}$ to denote the fundamental domain of the lattice without specifying the basis.

\begin{proposition}
    Let $\mathcal{L} \subset \R^n$ be a lattice of dimension $n$ and let $\mathcal{F}$ be a fundamental domain for $\mathcal{L}$. Then every vector $\bm{v} \in \R^n$ can be written in the form 
    $$\bm{v} = \bm{f} + \bm{t}$$
    for $\bm{f} \in \mathcal{F}$ and $\bm{t} \in \mathcal{L}$ both unique and associated to the original $\bm{v}$.
\end{proposition}

Equivalently, the space $\R^n$ is spanned exactly (without overlap) by shifting the fundamental domain by the vectors from our lattice.
$$ \R^n = \bigcup_{\bm{t} \in \mathcal{L}} \{\bm{f} : \bm{f} \in \mathcal{F} \}$$

\begin{remark}
    Sometimes the \textit{fundamental domain} is refered to as a parallelepiped or parallelotope and denoted by caligraphic $\mathcal{P}$. If we take a matrix $\bm{B}$ to represent our lattice $\mathcal{L}$, then $\mathcal{P}_{1/2}(\bm{B}) = \{\bm{x}\bm{B}, \bm{x} \in [-1/2, 1/2]^n \}$ can also represent the (shifted by a half) fundamental domain of $\mathcal{L}$ (like for example in \cite{gentry}).
\end{remark}

We will now present two results that give us an upper bound on the length of the shortest vector in a lattice. This will later on be useful to determine the security and/or correctness of our schemes. These theorems are due to Hermite (1822 - 1901) and Minkowski (1864 - 1909).

\begin{theorem}[Hermite's Theorem]
    Every lattice $\mathcal{L}$ of dimension $n$ contains a nonzero vector $v \in \mathcal{L}$ satisfying
    $$ \norm{v} \leq \sqrt{n} \det(\mathcal{L})^{\frac{1}{n}}.$$
\end{theorem}
\krzys{add minkowski's theorem}\\
We can also need some notion of the shortest possible vector of a lattice. 
\begin{definition}
	For a $n$-dimensional lattice $\Ll$, we denote its shortest nonzero vector (also called the \textit{minimum distance}) by $\lambda_1(\Ll)$. Formally
	\[ \lambda_1(\Ll) := \min_{0 \neq \bm{v} \in \Ll} ||\bm{v}|| \]
	More generally, for $1 \leq i \leq n$, the $i$th sucessive minimum of $\Ll$ is
	\[\lambda_i(\Ll) := \inf \{r : \Ll \text{ has $i$ linearly independent vectors of length at most } r \}. \]
\end{definition}
For example, in order to be able to go ``away'' from the lattice $\Ll$ ``out'' into the ambient space $\R^n$ and still be able to return to the original vector, we need to make sure the distance $d$ we travel is at most $\lambda_1(\Ll)/2$. This is for example a requirement in the \prob{BDD} problem where the decoding is \textit{promised}. For the exact definition see Section \ref{hardness}.
\subsection{Algebraic Number Theory}

Algebraic number theory is the study of \textit{number fields}, \textit{rings of integers} and \textit{finite fields}. In this section we will provide all the necessary background needed to understand and verify the results presented in the cryptographic schemes later in the text. Most results will be stated without proof however all of them can be found in the book \textbf{Number Fields} by \textbf{Daniel A. Marcus} \cite{algebra} after which this sections is modelled.

\subsubsection*{Number Fields}
A \textit{number field} is defined as a subfield of $\oQ$ having finite dimension as a vector space over the rationals $\Q$. The \textit{degree} of a number field $K$ is defined as the dimension of $K$ over $\Q$. There exists a monic irreducible polynomial\footnote{Recall that we call polynomial monic if its leading coefficient is 1. It is an irreducible polynomial if it is irreducible as an element of the polynomial ring $\Q[x]$.} $f \in \Q[x]$ such that $K \cong \Q[x]/\langle f \rangle$. In fact, every monic and irreducible polynomial in $\Q[x]$ defines a number field via such isomorphism.

\begin{definition}[Algebraic integer]
    An element $\alpha \in \C$ is an \textit{algebraic integer} if and only if, it is a root of some monic polynomial in $\Z[x]$.
\end{definition}
In fact, the set of \textit{algebraic integers} forms a ring.

\begin{definition}[Ring of Integers]
We define the \textit{ring of integers} (sometimes also called \textit{maximal order}) $\Oo_K$ of a number field $K$ as the intersection:
$$
  \Oo_K = K \cap \overline{\Z} = \{x \in K : \text{ $x$ is an algebraic integer}\}.
$$
\end{definition}

\begin{example}
    The field $K = \Q$ is a number field of degree 1. Its ring of integers is, as one can guess, the ordinary integers $\Z$.
\end{example}

\begin{example}
    The ring of Gaussian integers $\Z[\sqrt{-1}]$ is the ring of integers of $K = \Q(\sqrt{-1}) = \{a + b\sqrt{-1}\, :\, a,b\in \Q\}$ which has degree 2 since $x^2+1$ is the minimal (and irreducible) polynomial of $\sqrt{-1}$ over $\Q$.
\end{example}

As another example, we can make a following statement about the ring of integers of a quadratic extension of rationals (real quadratic field).
\begin{lemma}
     Let $d \in \Z$ be a square-free integer. For the field $K = \Q(\sqrt{d})$, its ring of integers is 
     \[ \Oo_K = 
	 \begin{cases} 
	     \Z[\sqrt{d}] & \text{if $d \equiv 2, 3 \mod 4$}, \\
	     \Z[(1 + \sqrt{d})/2] & \text{otherwise}.
     	 \end{cases}
     \]
\end{lemma}

\begin{proof}
	Take $d \equiv 1 \mod 4$ square-free.
\end{proof}

\begin{example}
	For $K = \Q(\sqrt{5})$ the ring of integers is $\Oo_K = \Z[(1 + \sqrt{5})/2]$. 
\end{example}

\subsubsection*{Cyclotomic fields}

\begin{definition}[Roots of unity]
    Given a field $K$ and a positive integer $n$, an element $\zeta \in K$ is called \textit{primitive $n$-th root of unity} if $\zeta$ has order $n$ in the multiplicative group $K^{\cross}$. (In other words, $\zeta^n = 1$ and $\zeta^m \neq 1$ for $1 \leq m < n$).
\end{definition}
The minimal polynomial $\Phi_n$ of $\zeta$ over $\Q$ is called the $n$-th cyclotomic polynomial. Formally we define it as 
\[ \Phi_n(x) = \prod_{\gcd(k,n) = 1} \bigg( x - e^{2\sqrt{-1} \pi k/n} \bigg) \]
It can be shown that $\zeta^i = \zeta^j$ if and only if $i = j$. The following equality is very useful for computing the polynomial itself:
\begin{equation}\label{comp_cycl} 
\Phi_n(x) = (x^n - 1) \bigg/ \prod_{d | n} \Phi_d(x) 
\end{equation}
\begin{example}
  Take $n = 8$. Then we can use equation \ref{comp_cycl} to compute:
\end{example}
As a usefull corollary of the equation \ref{comp_cycl}, we can prove that for $n$ a power of 2, $n = 2^k$ for some $k \geq 1$, the equation is of the form $x^{2^{k - 1}} + 1$.
\begin{corollary}
  Let $n= 2^k$ for some $k \in Z_{> 0}$. Then $\Phi_n(x) = x^{2^{k-1}} + 1$.
\end{corollary}
\begin{proof}
  By induction,
\end{proof}

What is worth noting here, is that for a number field $\Q(\alpha)$ for some $\alpha \in \C$, the ring of integers is not necessarily the $\Z[\alpha]$. Instead, $\Z[\alpha]$ is what's called an \textit{order} in $\Oo_K$. We will not consider them in general here because they are not relevant for our study. However, one very useful feature of cyclotomic fields is that their ring of integers is actually just $\Z[\zeta]$. That is - $\Oo_K = \Z[\zeta]$ for $K = \Q(\zeta)$ and $\zeta$ is some $n$-th root of unity. This greately simplifies the approach in proving some of the results later in this paper. In general, a field $K = \Q(\alpha)$ such that $\Oo_K = \Z[\alpha]$ is called a \textit{monogenic} field or a \textit{simple algebraic extension}. For more details on orders, look at for example Chapter 5 of \cite{stein}.

\begin{proposition}
	The ring of integers of a cyclotomic number field $K = \Q(\zeta)$ is $\Oo_K = \Z[\zeta]$.
\end{proposition}

\subsubsection*{Embeddings in $\C$}
Let $K = \Q(\alpha)$ be a number field of degree $n$ for some $\alpha$. Then there are exactly $n$ embeddings (injective ring homomorphisms) of $K$ in $\C$. These are easily described by observing that $\alpha$ can be sent to any one of its $n$ conjugates over $\Q$. Each conjugate $\beta$ determines a unique embedding $(\sigma_i: K \rightarrow \C$ and every embedding must arise in this way since $\alpha$ must be sent to one of its conjugates).

\begin{example}
    The quadratic field $\Q[\sqrt{d}]$, $d$ squarefree, has two embeddings in $\C$: The identity mapping, and also the one which sends $a + b\sqrt{d}$ to $a - b\sqrt{d}$ ($a$, $b$ $\in \Q$), since $\sqrt{d}$ and $-\sqrt{d}$ are the two conjugates of $\sqrt{d}$. The $n$-th cyclotomic field has $\varphi(n)$ embeddings in $\C$, the $\varphi(n)$ automorphisms where $\sigma_i(\zeta) = \zeta^i$.
\end{example}

\begin{definition}[Canonical embedding]

\end{definition}
\pinar{what do we want to consider as a cyclotomic poly? is $n=2^a$? is $n$ a prime? and why?}


\pinar{Maximal orders (ring of integers) are dedekind domains, embedding of $\Q(\alpha)$ to $\C$ hence embedding of the ideals. Properties of ideals in dedekind domains, operations, unique factorization and so on. all the necessary info. }

\subsection{Complexity Theory and Hard Problems} \label{hardness}
When talking about cryptography, we cannot avoid talking about algorithms as an encryption scheme is simply a instruction on how to encode sensitive data. We will thus require some terminology from computational complexity theory - a field on the overlap of computer science and mathematics. The problems it is concerned are those of the time and space required for solving computational problems. Most of the time, the goal of the study is to prove the lower bound on the resources required to solve a problem using the best know algorithm. For example, how long does it take to find a factorization of a large composite number. The goal of this section is to give an overview of few parts of the field that we will be using throughout the paper. 

\subsubsection*{}
We begin with necessary terminology. Throughout this section, $n$ shall denote the ``size'' of the input to the algorithm. Here the word size could mean many things like for example the bit-length of a number or its value in the base 10 representation.

In order to represent the time or space required to solve the problem, most of the time we will use the ``big-O'' notation. For any function $f$ and $g$ of some variable $x$, is defined as $f(x) \in O(g(x))$ if there exists some constant $M$ and $x_0$ such that $f(x) \leq Mg(x)$ for all $x \geq x_0$.

\pinar{will wait}
mention "efficient", "negligible", "hard" and define reductions. show some problems like svp, cvp, bdd
In this section we will briefly introduce what it means for a problem to be considered \textit{hard} and provide couple of examples
We define a \prob{negligible} amount in $n$ as an amount that is asymptotically smaller than $n^{-c}$ for any constant $c > 0$. More precisely, 
\begin{definition}\label{negl}
    $f (n)$ is a \prob{negligible} function in $n$ if $\lim_{n \to \infty}n^{-c} f (n) = 0$ for any $c > 0$.
\end{definition}
The best know examples
\prob{FACTORIZE}
\subsubsection*{Quantum computations}
I'm not sure if that is supposed to be in a section about preliminaries but I also don't want to include it in the introduction coz its a bit long

