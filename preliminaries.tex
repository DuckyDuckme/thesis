\subsection{Notation}
Most of the notation is "standard" in the field of number theory and cryptography. Nonetheless, to avoid any misunderstandings and simplify some statements, we will present the notation used throughout the text. Anything that is not mentioned here shall be defined "on the go" with definitions and such. \\

\noindent Any scalars $a, t, \beta, \dots$ are represented by non-bold, latin or greek letters.\\
Bold symbols $\bm{v}, \bm{x}, \bm{s}, \dots$ will denote vectors. \\
$\alg{Algorithms}$ will be represented by $\alg{lmodern font}$ names such as $\alg{Encrypt}$ or ${\fontfamily{lmss}\selectfont Evaluate}$.\\
Traditionally, the symbols $\Z, \Q, \R, \C$ shall represent the sets of integer, rational, real and complex numbers, respectivelly.

\subsection{Lattices}
\subsubsection*{Basic Definitions}
We define a \textit{lattice} as a discrete additive subgroup of $\R^n$. Once we fix a basis $\B = (\bm{b}_1, \dots ,\bm{b}_n) \in \R^n$ we can then describe the lattice as
$$ \Lambda = \mathcal{L}(\bm{B}) = \Bigl\{ \sum_i z_i \bm{b}_i : z_i \in \Z \Bigl\}.$$

There are many bases for a lattice (actually, for $n \geq 2$i, there are infinitely many as can be proven using a diagonalization argument), some "better" than others. This will be the foundation for some of the problems like SVP or CVP.

\begin{example}
    The simplest example of a lattice is the $\Z^n$ itself. Taking the standard basis $\B_1 = (\bm{e}_1, \dots, \bm{e}_n)$ we obtain
$$\mathcal{L}(\bm{B}_1) = \Bigl\{ \sum_i z_i \bm{e}_i : z_i \in \Z \Bigl\} = \Z^n.$$
\end{example}
More generally, $\Lambda$ is a lattice of rank $m$ in $\R^n$ if it is a rank $m$ free abelian group. Recall that we call a group \textit{free abelian group} of rank $m$ if it can be written as $\Lambda = \Z\beta_1 \oplus \cdots \oplus\Z\beta_m$ with $\beta_1, \dots, \beta_m$ linearly independent over $\R$ where $\oplus$ represents the direct sum. In this paper we will only consider lattices of full rank $n$. 

\begin{remark}
    We can also view the vectors $\bm{b}_i$ as the columns of the matrix $\B \in \R^n \cross \R^n$ in which case, our definition becomes:
    $$\Lambda = \mathcal{L}(B) = \{\bm{Bz} :  \bm{z} \in \Z^n \}.$$
\end{remark}

Reciprocally, any matrix $\bm{B} \in GL_n(\R)$ spans a lattice: the set of all integer linear combinations of its rows.

\begin{example}
\begin{enumerate}
    \item $\mathcal{L} = \begin{pmatrix}
        1 & 0\\
        0 & 1
	\end{pmatrix}$ in which case $\bm{b}_1 = \big(\begin{smallmatrix}
          1\\
          0
	\end{smallmatrix}\big)$ and $\bm{b}_2 = \big(\begin{smallmatrix}
          0\\
          1
        \end{smallmatrix}\big)$
    \item $\mathcal{L} = \{(z_1,z_2) : z_1 + z_2 \text{ is even \krzys{would odd work here?}}\}$
    \item $\mathcal{L} = \begin{pmatrix}
        13 & 21\\
        21 & 34
        \end{pmatrix}$
\end{enumerate}
\end{example}

As noted before, the basis of a lattice is not unique. There is one that is particularly interesting to us, namely, the \textit{Hermite Normal Form} (HNF). A basis $\B$ is in HNF if it is upper triangular, all elements on the diagonal are strictly positive and any other element $\bm{b}_{i,j}$ satisfies $0 \leq \bm{b}_{i,j} < \bm{b}_{i,i}$.

\subsubsection*{Fundamental Domain}
\begin{definition}[Fundamental Domain] \label{fundamental}
    Let $\mathcal{L}$ be a lattice of dimension $n$ and let $(\bm{b}_1, \dots, \bm{b}_n)$ be a basis for $\mathcal{L}$. The \textit{fundamental domain} (or \textit{fundamental parallelepiped}) for $\mathcal{L}$ corresponding to this basis is the set
    $$ \mathcal{F}(\bm{b}_1, \dots, \bm{b}_n) = \{t_1\bm{b}_1 + \cdots + t_n\bm{b}_n : 0 \leq t_i < 1 \}.$$
\end{definition}

We define the \textit{volume} of $\mathcal{F}(\bm{B})$ as the volume of the corresponding parallelepiped in $\R^n$. The \textit{volume} - closely connected to the determinant - plays a very important role in our study which will become evident in later chapters. One of the advantages, of defining the fundamental domain, is that we can formalize the area (or the determinant) or any given lattice. Recall that a lattice is just a countable collection of points and therefore has no volume by itself. This, however, is resolved by introducing the following.

\begin{definition}
    Let $\mathcal{L}$ be a lattice of dimension $n$ and let $\mathcal{F}(\bm{B})$ be a fundamental domain for $\mathcal{L}$ over some basis $\bm{B}$. We define the \textit{determinant} of that lattice as
    $$ \det (\mathcal{L}) = \VF{\bm{B}} = |\det (\bm{B}) |$$
\end{definition}

The next two propositions are \textit{de facto} foundation for lattice based cryptography. The first one states that the $\det (\mathcal{L})$ does not depend on the choice of the basis for that lattice. The second, that our whole ambient space $\R^n$ can be described using only vectors from the lattice and the fundamental domain. We will only give an outline of the proofs for the sake of keeping this section compact. Full proofs, however, can be found in \cite{book}, chapter 6.4.

\begin{proposition}
    The $\det (\mathcal{L})$ of an $n$-dimensional lattice is invariant under the choice of the basis.
\end{proposition}

\begin{proof}[Outline of the proof]
    Let $\bm{B}_1, \bm{B}_2$ be two bases for a lattice $\mathcal{L}$. The crucial part of the proof is to note that any two bases are related by some unimodular matrix $U$ (i.e. a matrix with the determinant of $\pm 1$) s.t. $\bm{B}_1 = U \bm{B}_2$. It now easily follows to compute $| \det (\bm{B}_1) | = \det (\mathcal{L}) = | \det (U \cdot \bm{B}_2) | = | \det(U) | \cdot | \det(\bm{B}_2) | = | \det(\bm{B}_2)|$ 
\end{proof}

From now on we will write $\mathcal{F}$ to denote the fundamental domain of the lattice without specifying the basis.

\begin{proposition}
    Let $\mathcal{L} \subset \R^n$ be a lattice of dimension $n$ and let $\mathcal{F}$ be a fundamental domain for $\mathcal{L}$. Then every vector $\bm{v} \in \R^n$ can be written in the form 
    $$\bm{v} = \bm{f} + \bm{t}$$
    for $\bm{f} \in \mathcal{F}$ and $\bm{t} \in \mathcal{L}$ both unique and associated to the original $\bm{v}$.
\end{proposition}

Equivalently, the space $\R^n$ is spanned exactly (without overlap) by shifting the fundamental domain by the vectors from our lattice.
$$ \R^n = \bigcup_{\bm{t} \in \mathcal{L}} \{\bm{f} : \bm{f} \in \mathcal{F} \}$$

\begin{remark}
    Sometimes the \textit{fundamental domain} is refered to as a parallelepiped or parallelotope and denoted by caligraphic $\mathcal{P}$. If we take a matrix $\bm{B}$ to represent our lattice $\mathcal{L}$, then $\mathcal{P}_{1/2}(\bm{B}) = \{\bm{x}\bm{B}, \bm{x} \in [-1/2, 1/2]^n \}$ can also represent the (shifted by a half) fundamental domain of $\mathcal{L}$ (like for example in \cite{gentry}).
\end{remark}

We will now present two results that give us an upper bound on the length of the shortest vector in a lattice. This will later on be useful to determine the security of our schemes. These theorems are due to Hermite (1822 - 1901) and Minkowski (1864 - 1909).

\begin{theorem}[Hermite's Theorem]
    Every lattice $\mathcal{L}$ of dimension $n$ contains a nonzero vector $v \in \mathcal{L}$ satisfying
    $$ \norm{v} \leq \sqrt{n} \det(\mathcal{L})^{\frac{1}{n}}.$$
\end{theorem}

\begin{remark}
    
\end{remark}

\subsection{Algebraic Number Theory}

Algebraic number theory is the study of \textit{number fields}, \textit{rings of integers} and \textit{finite fields}. In this section we will provide all the necessary background needed to understand and verify the results presented in the cryptographic schemes later in the text. Most results will be stated without proof however all of them can be found in the book \textbf{Number Fields} by \textbf{Daniel A. Marcus} \cite{algebra} after which this sections is modelled.

\subsubsection*{Number Fields}
In general, a \textit{number field} is defined as a subfield of $\oQ$ having finite degree (the dimension as a vector space) over the rationals $\Q$. Throughout this section, we fix the imaginary numbers $\oQ = \C$ as an algebraic closure of the rationals.

\pinar{Degree of a field over $\Q$. Define the monic minimal polynomial.}

\begin{definition}[Algebraic integer]
    An element $\alpha \in \C$ is an \textit{algebraic integer} if it is a root of some monic polynomial with coefficients in $\Z$.
\end{definition}

In fact, the set of algebraic integers forms a ring (under the usual addition and multiplication operations in $K$). To see this, we make use of the following lemma \krzys{this seems like unnecessarily addition, can't this be proven more simply?}.

\begin{lemma} \label{lemma1}
    Any $\alpha \in \C$ is an algebraic integer if and only if, the additive group of the ring $\Z[\alpha]$ is finitely generated.
\end{lemma}
\begin{proof}
    ($\Rightarrow$): If $\alpha$ is a root of a monic polynomial over $\Z$ of degree $n$, then the additive group $\Z[\alpha]$ is generated by $1, \alpha, \cdots, \alpha^{n-1}$. \\
    ($\Leftarrow$): \krzys{TODO}
\end{proof}

\begin{definition}[Ring of Integers]
We define the \textit{ring of integers} (sometimes also called \textit{number ring}) $\Oo_K$ of a number field $K$ as the largest ring in the intersection:
$$
  \Oo_K = K \cap \overline{\Z} = \{x \in K : \text{ $x$ is an algebraic integer}\}.
$$
The \textit{degree} of the number field is defined as the degree of the 
\end{definition}

\begin{example}
    The field $K = \Q$ is a number field of degree 1.   
\end{example}

\begin{theorem}
     In general, for a quadratic field $K = \Q(\sqrt{d})$ with square-free $d \in \Z$, $\Oo_K$ is $\Z[\sqrt{d}]$ or $\Z[(1 + \sqrt{d})/2]$ depending on $d \mod 4$.
\end{theorem}

\begin{example}
     Its ring of integers $\Oo_K$ are the ordinary integers $\Z$. The field 
    $$
    K = \Q(\sqrt{-1}) = \{a + b\sqrt{-1}\, \mid\, a,b\in \Q\}
    $$ 
    has degree 2 since $x^2+1$ is the minimal polynomial of $\sqrt{-1}$ over $\Q$ and with $\Oo_K = \Z[\sqrt{-1}]$. 
    
    for $K = \Q(\sqrt{5})$ the ring of integers is $\Oo_K = \Z[(1 + \sqrt{5})/2]$. 
\end{example}

\subsubsection*{Cyclotomic fields}

\begin{definition}[Roots of unity]
    Given a field $K$ and a positive integer $n$, an element $\zeta \in K$ is called \textit{primitive n-th root of unity} if $\zeta$ has order $n$ in the multiplicative group $K^{\cross}$. (In other words, $\zeta^n = 1$ and $\zeta^m \neq 1$ for $1 \leq m < n$).
\end{definition}

The minimal polynomial $\Phi_n$ of $\zeta$ over $\Q$ is called the $n$-th cyclotomic polynomial.

\subsubsection*{Embeddings in $\C$}
Let $K = \Q(\alpha)$ be a number field of degree $n$ for some $\alpha$. Then there are exactly $n$ canonical embeddings (injective ring homomorphisms) of $K$ in $\C$. These are easily described by observing that $\alpha$ can be sent to any one of its $n$ conjugates over $\Q$. Each conjugate $\beta$ determines a unique embedding $(\sigma_i: K \rightarrow \C$ and every embedding must arise in this way since $\alpha$ must be sent to one of its conjugates).

\begin{example}
    The quadratic field $\Q[\sqrt{d}]$, $d$ squarefree, has two embeddings in $\C$: The identity mapping, and also the one which sends $a + b\sqrt{d}$ to $a - b\sqrt{d}$ ($a$, $b$ $\in \Q$), since $\sqrt{d}$ and $-\sqrt{d}$ are the two conjugates of $\sqrt{d}$. The $n$-th cyclotomic field has $\varphi(n)$ embeddings in $\C$, the $\varphi(n)$ automorphisms where $\sigma_i(\zeta) = \zeta^i$.
\end{example}

\iffalse
\begin{corollary}
    Let $n = 2^r$ for some $r \in \Z$. The ideals in the ring of integers of the cyclotomic field of $2n$-th roots of unity $K = \Q[2n]$ are $n$-dimensional lattices.
\end{corollary}

\begin{proof}
    Note that the ideals have degree $\varphi(2n) = \varphi(2^{r+1}) = 2^{r+1} - 2^r = 2^r = n$. The rest follows by the previous theorem.
\end{proof}
\fi

\pinar{what do we want to consider as a cyclotomic poly? is $n=2^a$? is $n$ a prime? and why?}


\pinar{Maximal orders (ring of integers) are dedekind domains, embedding of $\Q(\alpha)$ to $\C$ hence embedding of the ideals. Properties of ideals in dedekind domains, operations, unique factorization and so on. all the necessary info. }

\subsection{Complexity Theory and hard problems} 
\label{hardness}
\pinar{will wait}
In this section we will briefly introduce what it means for a problem to be considered \textit{hard} and provide couple of examples

The best know examples
\prob{FACTORIZE}
\subsubsection{Shor's Algorithm}
I'm not sure if that is supposed to be in a section about preliminaries but I also don't want to include it in the introduction coz its a bit long

