\subsection{Actual introduction here with some funny name}
In recent years, the topic of lattice based cryptography (LBC) has seen many developments towards an efficient and quantumly secure basis for cryptographic constructions. Starting with Ajtai's seminal work \cite{ajtai} on one-way functions that can be reduced to  as one of the main contenders for an efficient and resistant to quantum computers scheme.
\subsection{Motivation}
include the table from page 16 of \cite{bernstein} of systems broken by quantum computers\\
On the $18^{th}$ of November 2022, a \href{https://www.whitehouse.gov/wp-content/uploads/2022/11/M-23-02-M-Memo-on-Migrating-to-Post-Quantum-Cryptography.pdf}{document} was issued on migrating to post-quantum cryptography.

\iffalse
This bachelor thesis can be seen as a survey on the recent developments in the LBC and improvements in fully homomorphic cryptographic schemes. Nonetheless, there are few minor contributions that we note here.

Firstly, and most importantly, the introduction of many examples that we hope will be of great help for anyone trying to learn and understand LBC schemes. These include comptutations as well as images and tables that help visualize some key concepts related to Gaussian samples or lattices.

Secondly, we present and explicitly state few results that are otherwise left simply as claims or propositions. These mostly include parts on the algebraic number theory for the Section \ref{ring-lwe} on ring-LWE.
\fi

\subsection{Outline}
The sections in this paper are to be read consecutively with one exception. The Section \ref{ant} on algebraic number theory is mostly useful to prove concepts for ring-LWE from Section \ref{ring-lwe}. If one is not interested in the proofs from this section, it is possible to skip those two since the idea behind ring-LWE is almost identical to the LWE with just minor changes that are not crucial to understand other parts of this paper.

\subsection*{Acknowledgements}
I would like to thank my supervisors, Pınar Kılıçer, for her time and guidance through the areas of algebra and algebraic number theory, as well as Marcello Seri for the project idea and various work-aiding tools. Lastly, I would like to thank my parents and in particular my dad who, long time ago, said ``\textit{You will be thanking me for that later.}'' when urging me to do my mathematics homework. I think this is the right place to say, thank you dad.
