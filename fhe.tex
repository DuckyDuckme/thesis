Fully Homomorphic Encryption (FHE) has been referred as the ``holy grail'' of modern cryptography as it was one of the most sought goals for the past couple of decades. First formally introduced by Rivest, Adleman and Dertouzos in \cite{primal} (at the time called ``privacy homomorphism''), shortly after the discovery of public key cryptography, it has been an open and elusive problem. Only ``recently'', in~2009, Craig Gentry proposed first FHE in his PhD thesis \cite{gentry_phd}. Since then, there has been a lot of development in the area like for example \krzys{TODO: finish developments of fhe}.

Simply stated, in homomorphic encryption we want our data to be secure but we also want to perform calculations on it. This is useful when we need a third party (e.g. someone with more computational power) to perform operations on our data while still retaining privacy. Alice can store her data somewhere on external server (for example the cloud) and ask to perform computations on it. We can for example query searches without the engine knowing what is actually being searched for.

In other words, we would like our encryption scheme -- call it $\E$ -- to satisfy the following. Say the ciphertexts $c_i$'s decrypt to messages $m_i$'s. Then we want
\[ \alg{Decrypt}_{\E}(c_1 + c_2) = m_1 + m_2, \qquad \alg{Decrypt}_{\E}(c_1*c_2) = m_1*m_2 \]
Equivalently, we want $\alg{Decrypt}$ to be a ring homomorphism. $\E$ being \textit{fully homomorphic} means that whenever $f$ is a composition of \textbf{arbitrily many} additions and multiplications, then $\alg{Decrypt}_{\E}(f(c_1, \dots, c_n)) = f(m_1, \dots, m_n)$\footnote{There are two more technical requitements, namely \textit{compactness of the ciphertexts} and \textit{efficiency} but we will not consider them in this paper.} which is also refered to as the \textit{correctness} of the scheme.

\begin{remark} \label{algs}
    Typically, an encryption scheme $\mathcal{E}$ is a tuple of $\alg{KeyGen}_{\E}$, $\alg{Encrypt}_{\E}$ and $\alg{Decrypt}_{\E}$ (representing the key-generation, encryption and decryption respectively), all of which we require to be \textit{efficient} -- i.e. run in time poly($\lambda$) - polynomial in the security parameter $\lambda$ that represents the bit-length of the keys (see for example \cite{katz} or \cite{book} for more details on the abstract build of a encryption scheme). A homomorphic encryption scheme has a fourth algorithm -- $\alg{Evaluate}_{\E}$ which we associate with some set of \textit{permitted functions}. In our case this will simply be $\alg{Add}_{\E}$ and $\alg{Mult}_{\E}$ which we will introduce in further sections. Adopting the notation from \cite{easy_fhe} we will denote by $\mathcal{F}_{\E}$, the generalized set of such functions.
\end{remark}

One might ask a question now, how secure can such scheme ultimately be? After all, we are giving the adversary a quite powerful tool in the form of being able to compute (ultimately) \textit{any} function on our data.

\subsection{Somewhat Homomorphic Encryption}\label{int_she}
Before we introduce the solution on to how to construct such FHE presented by Gentry, we will start with something slightly simpler, introduced in \cite{int_scheme} by van Dijk et al. Their scheme works over the integers rather than lattices but relies on a similar assumption. Namely, that finding the greatest common divisor of many ``noisy'' multiples of a number is computationally difficult. We will come back to this problem later. To keep the exposition compact, we will avoid specifying most parameter choices.

\subsubsection*{Symmetric Key Scheme}
We begin with the symmetric key scheme. We take our message to be a bit $m \in \{0,1\}$. The private key is an odd integer $p$ (no necessarily prime). To encrypt our message $m$, we choose integers $q$ and $r$ at random (such that the magnitude of $2r$ is smaller than $p/2$). We obtain the ciphertext $c$ by computing: 
\begin{equation}c = pq + 2r + m.\end{equation}
If we now want to decrypt our message, simply compute $(c \mod p) \mod 2$. \\
Let's say we have two messages $c_1$ and $c_2$. Then we can compute:
$$ c_1 + c_2 = m_1 + m_2 + 2(r_1 + r_2) + p(q_1 + q_2),$$
$$ c_1 * c_2 = m_1 * m_2 + 2(m_1r_2 + m_2r_1 + 2r_1r_2) + p(m_1q_2 + m_2q_1 + 2(r_1q_2 + r_2q_1) + pq_1q_2)$$
where we can see that the noise grows with each operation and the message becomes impossible to decrypt after we do too many of them. If we can assure that $2(m_1r_2 + m_2r_1 + 2r_1r_2)$ is small enough - i.e. smaller than $p$\footnote{When $2r > p$ then it might be the case that $2r = 1 \mod p$ and so $pq + 2r + m \mod p = 1 + m \neq m$.} - then we can assure that $\alg{Decrypt}(c_1 * c_2)$ evaluates correctly to the starting $m_1 * m_2$. Notice that $\alg{Decrypt}$ removes all the noise. This will be useful later for ``bootstrapping'' - a term introduced later on.

This simple encryption scheme is thus somewhat homomorphic as per definition by Gentry in \cite{gentry_phd} – namely, it can be used to evaluate low-degree polynomials over encrypted data. Further on in the Section 6 of \cite{int_scheme}, van Dijk et al. use the techniques (called bootstrapping and squashing) to lift it to a Fully Homomorphic Scheme.
\subsubsection*{Public Key Scheme}
The public key scheme is build very similarly. The private key $p$ stays the same. For the public key, sample $x_i = p q_i + 2r_i$ for $i = 0, 1, \dots, t$ where the $q_i$ and $r_i$ stay as before. The $x_i$ may be viewed as encryption of 0 under the symmetric key scheme. The $x_i$ are now taken s.t. $x_0$ is the largest, odd and $x_0 \mod p$ is even.

To now encrypt a message $m \in \{0,1\}$, chose a random subset $S \subseteq \{1, 2, \dots, t\}$ and a random integer $r$, and output
\begin{equation} c = (m + 2r + 2\sum_{i \in S} x_i) \mod x_0.\end{equation}
To decrypt, we again output $m = (c \mod p) \mod 2$.

The security of this preliminary SH scheme relies on the \textit{Approximate GCD Problem}\footnote{Later in \cite{revisited}, a reduction was constructed to LWE. This means, that under few more assumptions, this problem (and by extension any scheme based on it) is as secure as one based on LWE.}. In the simplest case, Euclid has shown us, that given two integers $c_1$ and $c_2$, it is easy to compute their $\gcd$. However, suppose now that $c_1 = p \cdot q_1 + r_1$ and $c_2 = p \cdot q_2 + r_2$ are ``near'' multiples of $p$, where $r_1$ and $r_2$ is some small noise sampled at random. This turns out to be much more difficult. In fact, if we pick our values appropriately (see \cite{easy_fhe} Section 3.4 and \cite{int_scheme} Section 3 for details) we do not know any efficient (running in polynomial time) algorithm even if we are given arbitrarily many samples $c_i = r_i + p \cdot q_i$.

However, this comfortable security comes at great cost because, as shown in \cite{int_scheme}, the parameters chosen to assure the secrecy, yield a scheme that has complexity of $\tilde{O}(\lambda^{10})$ where $\lambda$ is our security parameter (the greater it is the more secure message). As a small example, consider $\lambda = 10$ as the (small) key size. To now encrypt a single(!) bit, it will take approximately $10^{10}$ operations. On a modern laptop this would take a little less than 5 seconds. To send the message `hello', we need to use 5 letters $\cdot$ 16-bits per letter $= 5\cdot 16\cdot5 = 650$ seconds which is almost 11 minutes! As one can imagine, this is completely impractical for most applications.

\subsection{Fully Homomorphic Encryption}
We will now present the main idea introduced in Gentry's PhD thesis \cite{gentry_phd}. Namely, what is bootstrapping, why do we need it and how does it work. % how can we make use of ideal lattices to create a encryption scheme that can handle an arbitrary amount of functions. We will proceed in parallel with the original by introducing the idea of ``bootstrapping'', through a construction of a suitable SH scheme (this is where the ideal lattices show up) and finishing with the ``squashing''.

\subsubsection{Boostrapping}
We are faced with a problem. Because our method relies on some error being added to the message, it builds up after we perform operations on our data. The scheme $\E$ can handle functions in a limited set $\mathcal{F}_{\E}$ until the noise becomes too large. Is there a way for us to somehow expand this set and yet retain the homomorphic properties of the scheme? Can we, further on, expand this set to include an arbitrary polynomial function? The answer turns out to be yes. As shown by Gentry, one of the requirements for this is that the scheme can decrypt its encryption correctly ``and some''. A bit more formally, we require that the $\alg{Decrypt}_{\E} \in \mathcal{F}_{\E}$. This part will be a brief explanation of what the bootstrapping is and how we can achieve it using (also introduced in the same paper) ``squashing''.

\begin{remark}
  In cryptography, which is a field in the intersection of mathematics and computing science, instead of general functions, one often considers a \textit{circuit} instead. Roughly speaking, a circut is a translation of what a mathematician thinks when they say ``function''. It consists of (finite amount) of gates which are just boolean functions. For example there is a $\alg{NOT}$ gate that takes a bit $b \in \{0,1\}$ and outputs $b + 1$, where all the operations are performed modulo 2. Others include for example $\alg{XOR}$ - this is an exclusive logical $\alg{OR}$ - or $\alg{NAND}$ which is a $\alg{AND}$ followed up by a $\alg{NOT}$ gate. From a theoretical point of view, one can use solely a $\alg{NAND}$ gate to represent any circuit and thus we will be mostly concerned with that one.
\end{remark}

Imagine we have a SH scheme $\E$ (one can think of the one from previous section as a concrete example) that is correct for its own decryption circuit augumented by an $\alg{NAND}$ gate. We call such scheme \textit{bootstrappable}. As shown in the paper, one can use this circuit to create a (leveled) fully homomorphic encryption $\E^{(d)}$. It is called \textit{leveled} because the correctnes depends on the ``depth'' (or the level) $d$ of the circuit making it depend on $d$. It can be show that by appropriately augumenting the public key, such leveled scheme can be made independent of depth and thus made into a \textit{fully homomorphic} one. Moreover, if the scheme itself is secure\footnote{The precise security mentioned is the semantic security against chosen plaintext attacks. One can think of it as requiring that the same message $m$ has different outputs $c$ on different runs of the same encryption algorithm (it is non-deterministic). See \cite{katz} or \cite{lattice-survey} for a more precise definition.}, then so is any $\alg{Evaluate}_{\E^(d)}$ algorithm. All of this is captured in the following theorem.

\begin{theorem}[\cite{gentry}]\label{boot}
  One can construct a (semantically secure) family $\{ \E^{(d)} \}$ of leveled fully homomorphic encryption schemes from any (semantically secure) bootstrappable encryption scheme $\E$.
\end{theorem}

Why is this the correct requirement for a FHE? Suppose that there is an ``error'' associated with each ciphertext, just like in the scheme from Section \ref{int_she}. Then, as also noted there, after we perform one operations too many, the error builds up soo much that we are no longer able to decrypt correctly. We would therefore like to somehow make the error small enough again and ``refresh'' the ciphertext without using the secret key. Clearly we could get rid of the error completely if we were to decrypt it and create a ``fresh'' ciphertext of the same message. This is the precisely the idea, to decrypt the message but do it homomorphically! We obtain a homomorphic encryption by encrypting homomorphically -- we are bootstrapping. Gentry's idea on how to do it, was by appending the encryption of the secret key in the public key itself.

\begin{remark}
Note that this requires another assumption called ``circular security'' or KDM (Key Dependent Message) security. That is, we are assuming that the publication of the encryption of a secret key does not leak any valuable information about the key itself. This is however very difficult to prove in practice but also no known attacks are know and hence it is just assumed along.
\end{remark}

One last question we have is, why do we even need the error? Cannot we create a \textit{semantically} secure encryption scheme that does not depend on any ``error'' to hide the message, and consequently, create a FHE without any sort of bootstrapping? It turns out that it is in fact not possible. Before we present the reason for this, let us motivate why we care about the semantic security so much in the case of HE.

\begin{example}
	Using cloud computing as an example, imagine that after we stored few files in the cloud, we want to retrive files from the cloud that contain the \textit{Answer to the Ultimate Question of Life, the Universe, and Everything}. We thus homomorphically query the engine for the file number 42. But can't the cloud just ``notice'' that the encryption of the file it is sending us now is the same as the encryption of the \textit{Answer to the Ultimate Question of Life, the Universe, and Everything}? It cannot, if our encryption is semantically secure, because in that case, it is difficult to tell if two ciphertexts encrypt the same message.
\end{example}

Let us now get back to why we need the error. For any \textit{deterministic} algorithm (like \alg{AES} or \alg{RSA}) that encrypts $m$ as $c$ every time it is run, any adversary can check if any $m_0$ encrypts as $c$ and compare the two. Thus, by definition, such scheme cannot be semantically secure. We thus need a \textit{probabilistic} algorithm that gives us different output every time it is run. This implies some sort of randomizing the ciphertext which can only be done (to our understanding) using some error to mask the message. Therefore we need some sort of ``bootstrapping'' (which is not necessarily the one mentioned above, this is only one way to achieve FHE but there could be more) to create a FHE.

\paragraph{How to bootstrap}
We will not present the precise idea behind \textit{bootstrapping}. To this end, recall that our generic homomorphic scheme $\E$ consists of four algorithms. These are:
\begin{enumerate}
	\item $\alg{KeyGen}_{\E}$ which generates the public and secret keys -- pk and sk respectively.
	\item $\alg{Encrypt}_{\E}(\text{pk}, m)$ which encrypts the message $m$ from the plaintext space $\mathcal{P}$ under the public key $\text{pk}$ and outputs the ciphertext $c$.
	\item $\alg{Decrypt}_{\E}(\text{sk}, c)$ which decrypts the ciphertext $c$ using the secret key $\text{sk}$.
	\item $\alg{Evaluate}_{\E}(\text{pk}, f, c_1, \ldots, c_t)$ which evaluates the function $f \in \mathcal{F}_{\E}$ from the set of permitted functions on the ciphertexts $c_1, \ldots, c_t$ using the public key pk.
\end{enumerate}
We shall now construct a new algorithm which we will call $\alg{Recrypt}_{\E}$ using only those four above. For simplicity, we assume that our plaintext space is just $\{0,1\}$. Imagine we have a valid ciphertext $c_1$ that encrypts $m_1$ under the public key pk$_1$. In order to decrypt it homomorphically, assume additionally that we have the secret key sk$_1$ encrypted under another public key pk$_2$ -- denote this encryption by $\hat{\text{sk}}_1$ (\krzys{idk how to make it span the whole variable}). Define now the following algorithm:\\
$\alg{Recrypt}_{\E}(\text{pk}_2, \alg{Decrypt}_{\E}, \bar{\text{sk}}_1, c_1)$:


\begin{align*}
	\text{Set } \hat{c}_1 & \leftarrow \alg{Encrypt}_{\E} (\text{pk}_2, c_1) \\
	\text{Set } c_2 & \leftarrow \alg{Evaluate}_{\E} (\text{pk}_2, \alg{Decrypt}_{\E}, \hat{\text{sk}}_1, \hat{c}_1) \\
	\text{Return } c_2 &
\end{align*}

Here we use the encryption of our secret key to evaluate the decryption function homomorphically under a new public key. Now we can see why we need the assumption that the scheme is KDM secure. Since the encryption of sk$_1$ is public, we need to make sure it does not leak any information about it. The statement of the Theorem \ref{boot} should now be intuitively clear. As long as the scheme $\E$ can evaluate its own decryption function, and the $\alg{Recrypt}$ algorithm leaves the new message with less error than we began with, we can use $\E$ to construct a leveled FHE scheme. In order to now evaluate \textit{any} function on our data, we would also like our scheme to correctly evaluate a $\alg{NAND}$ gate (recall that we can create any gate using a finite combination of $\alg{NAND}$ gates) in order to make some progress before the error ``blows up''.

\subsubsection{Simple lattice based scheme}
The motivation for the choice of lattices as opposed to number theoretic constructions (like RSA or ElGamma for example, which are based on exponentiation), is that the former has few desirable properties. Firstly, the lattices have lower decryption complexity and are therefore more suitable as a bootstrappable encryption scheme. Next, one requires not only one supported homomorphism like addition but also the multiplication -- lattices as ideals poses both. Those factors among many others, (see \cite{gentry} for more details) have led to the choice of \textit{ideal lattices}.

The idea behind the encryption is somewhat similar to the one of GGH. That is, we are going to fix a basis for a lattice (which is an ideal), publish the bad basis as the public key and keep the good basis as a secret key. The security of this scheme is based on the Ideal Coset Problem\footnote{Later, in 2010, Gentry showed that the scheme can be based on the worst-case \prob{BDD} problem -- \cite{worst-case}.} (\prob{ICP}) introduced below. Briefly stated, given an element of some ring $t \in R$, the \prob{ICP} asks us to distinguish between a representative of a coset taken from some distribution over an arbitrary ring and a uniform sample.

For a ring $R$ and a basis\footnote{By the word ``basis'' we simply mean the set of generators for the ideal but we use that instead to highlight the connection between the ideals and lattices as well as keep it consistent with Gentry's work.} $\B_{\I}$ for an ideal $\I \subset R$, we use the notation $R \mod \B_{\I}$ to represent the set of $r + \I$ for $r \in R$.

\begin{definition}[Ideal Coset Problem]
	Fix a ring $R$ along with an ideal $\I \subseteq R$ and sample an element $r \in R$ given a sampling algorithm \alg{Samp}. Now pick a basis $\B_{\I}$ for the ideal $\I$. Given the basis $\B_{\I}$, the challenger is asked to distinguish between $t \equiv r \mod \B_{\I}$ and $t$ being chosen uniformly.
\end{definition}
Few remarks are in place now. Firstly, this definition hides few details in favour of clearer notation. As one, we actually fix some ideal $\J \subseteq R$ such that $\I$ and $\J$ are coprime (i.e. $\I + \J = R$) first, and then, with respect to such $\J$, we instantiate the $\I$. This will become evident later in this section why this is required. Secondly, the procedure somewhat depends on the efficiency of those choices. One example would be how to pick such coprime ideals as well as how to sample the $r$ from the ring itself. The questions (partially) are answered in the paper \cite{gentry_phd} itself.

From now on, let $R$ be an arbitrary ring, and set $\I, \J \subset R$ as two coprime ideals of $R$.  We will also be using a sampling algorithm which we call \alg{Samp}$(x, \B_{\I} , R, \B_{\J})$ that samples from the coset $x + \I$. Let us now introduce the scheme in terms of rings and ideals only.

\begin{mdframed}
	$\alg{KeyGen}(\B_{\I}, R$) \\
	Generate two basis for $\J$, $\B_{\J}^{sk}$ and $\B_{\J}^{pk}$. The \textbf{public key} $pk$ includes $R$, $\B_{\I}$, $\B_{\J}^{pk}$ and the sampling algorithm \alg{Samp}. The \textbf{secret key} $sk$  also includes the $\B_{\J}^{sk}$. We denote by $\mathcal{P}$ the plaintext space which is (a subset of) $R \mod \B_{\I}$.\\
	$\alg{Encrypt}(\text{pk}, m)$ \\
	For a message $m \in \mathcal{P}$, set $c' \leftarrow$ \alg{Samp}$(m, \B_{\I}, R, \B_{\J}^{pk})$ and output $c \equiv c' \mod \B_{\J}^{pk}$.\\
	$\alg{Decrypt}(\text{sk}, c)$ \\
	For the ciphertext $c$, output $(c \mod \B_{\J}^{sk}) \mod \B_{\I}$.\\
	$\alg{Evaluate}(\text{pk}, f, C)$ \\
	This algorithm takes a set of ciphertexts $C = (c_1, \ldots, c_t)$ and applies the function $f \in \mathcal{F}_{\E}$ using the public key pk. It outputs $f(c_1, \ldots, c_t)$. Since we only support two operations from the ring, $f: R \rightarrow R$ needs to be a homomorphism.
\end{mdframed}

Having the bootstrapping in mind, we should pay close attention to the correctness of the decryption algorithm. Before proving it, let us state few definitions first.

\begin{definition}
	Let $X_{Enc}$ be the image of $\alg{Samp}$. Notice that all ciphertexts output by $\alg{Encrypt}$ are in $X_{Enc} + J$. Let $X_{Dec}$ equal $R \mod \B^{sk}_{\J}$, the distinguished representatives of cosets of $\J$ wrt the secret basis $\B^{sk}_{\J}$.
\end{definition}

\begin{definition}[Permitted Functions]
	Let
	\[ \mathcal{F}_{\E} := \{ f : \forall (x_1, \ldots, x_t) \in X_{Enc}^T, f(x_1, \ldots, x_t) \in X_{Dec}. \} \]
	In other words, $\mathcal{F}_{\E}$ is the set of those functions of which output will always be in $X_{Dec}$ when the input is from $X_{End}$.
\end{definition}
Aditionally, we say the $\hat{c}$ a \textit{valid ciphertext} if it equals $\alg{Evaluate}$(pk, $f, (c_1, \ldots, c_t)$ for some function $f \in \mathcal{F}_{\E}$. We can now state and prove the correctness of our scheme.
\begin{theorem}[Correctness]
	Assume $\mathcal{F}_{\E}$ is a set of permitted functions containing the identity. $\E$ is correct for $\mathcal{F}_{\E}$ –- i.e., $\alg{Decrypt}$ correctly decrypts valid ciphertexts.
\end{theorem}
\begin{proof}
	For $C = \{c_1, \ldots, c_t\}, c_k = m_k +i_k +j_k$, where $m_k \in \mathcal{P}, i_k \in \I, j_k \in \J$ and $m_k +i_k \in X_{Enc}$, we have
	\begin{align*}
		\alg{Evaluate}(\text{pk}, f, C) = & f(C) \mod \B_{\J}^{\text{sk}} \\
		= & f(m_1 +i_1 +j_1, \ldots, m_t+i_t+j_t) \mod \B_{\J}^{\text{sk}} \\	
		\in & f(m_1 +i_1, \ldots, m_t +i_t) + \J.
	\end{align*}
	If $f \in \mathcal{F}_{\E}$, then we have $f(X_{Enc}, \ldots, X_{Enc}) \in X_{Dec}$ and so
	\begin{align*}
		\alg{Decrypt} & (\text{sk}, \alg{Evaluate}(\text{pk}, C)) \\
		& = f(m_1+i_1, \ldots, m_t +i_t) \mod \B_{\I} \\
		& = f(m_1, \ldots, m_t) \mod \B_{\I} \\
		& = f(m_1, \ldots, m_t)
	\end{align*}
	as required.
\end{proof}

Once we have the correctness, we can focus on the actual implementations of any fully homomorphic schemes. This in turn implies that we should pay very close attention to the size of $X_{Enc}$ and $X_{Dec}$. Namely, we want to maximize $X_{Dec}$ and minimize $X_{Enc}$ (while still retaining security). \krzys{finish the geometric interpretation of $X_{Enc}$ and $X_{Dec}$ if there is time}.

\subsection{Further developments in FHE}
Since the inception of FHE, many new ideas have emerged as potential replacement for the details in the implementation in Gentry's work. In this section, we will briefly introduce some of such techniques and ideas. Finally the work we have done on LWE and its ring equivalent will pay off as we can rely on those results to abstractly create FHE scheme \krzys{i dont like this sentence but idk how to make it prettier}. 

\subsubsection{FHE from the standard LWE}
 One of such ideas is to replace the ideal lattices by an LWE sample instead. Let us now attempt to construct a SH scheme that is using the LWE assumption along the ideas presented in \cite{fhe-lwe}. Recall that a LWE$_{q, \chi}$ is specified by the odd modulus $q > 2$ and error distribution $\chi$. It provides us with the LWE distribution that we called $A_{s, \chi}$ where $\bm{s}$ was a random vector representing the secret key outputting us samples of the form
\[A_{s, \chi} \rightarrow (\bm{a}, b) = (\bm{a}, \langle \bm{a}, \bm{s} \rangle + e),\]
where $e$ was drawn from some distribution $\chi$ (usually taken to be the discrete Gaussian distribution with small standard deviation) over $\Z_q$ for $q$ prime\footnote{Since we will be using the decision version of the problem, we need to assume $q$ to be prime.}. We can now imagine a scheme that works as follows: to encrypt a single bit $m \in \{0,1\}$ using a secret key $\bm{s}$, draw a sample from $A_{s, \chi}$ and output the ciphertext
\[ c \; = \; (\bm{a}, b = \langle \bm{a}, \bm{s} \rangle +2e +m) \; \; \; \in \Z_q^n \cross \Z_q \]
Note that we have replaced the error $e$ with its two-times multiple in contrast to the original formulation by Regev as can be seen in Table \ref{lwe-enc}. This is not a problem at all because 2 and $q$ are coprime. To decrypt the message, first compute $\langle \bm{a},\bm{s} \rangle$ and substract that from $b$, giving $2e + m \mod q$ which, since $e \ll q$, is actually equal to $2e +m$ exactly. Finally, reduce modulo 2 and we are left with the original message $m$.

The scheme is clearly additively homomorphic. The issue arises when we try to multiply two ciphertexts. As shown in \cite{one-mult}, the (slight variation of) this scheme supports only a \textit{single} homomorphic multiplication with the expense of huge blowup in the ciphertext size.

To see how this is the case, consider a symbolic function $f_{\bm{a},b} : \Z_q^n \rightarrow \Z_q$ defined as:
\[f_{\bm{a}, b}(\bm{x}) = b - \langle \bm{a, x} \rangle \mod q = b - \bm{a}_i \bm{x}^i \; \; \; \in \Z_q \]
where we are using Eisenstein notation to represent elements $\bm{x} \in \Z_q^n$ (the transpose of the first element is implicit). Note now that the decryption is nothing else but evaluating this function on the secret key $\bm{s}$ and taking the result modulo 2. We can now define addition and multiplication using this function. Addition is straightforward, $f_{\bm{a}, b}$ is a linear function and so sum of two linear functions is still linear. Symbolically, $f_{\bm{a},b}(\bm{x}) + f_{\bm{a}',b'}(\bm{x}) = f_{(\bm{a}+\bm{a}',b+b')}(\bm{x})$ will represent the homomorphically added ciphertext $(\bm{a}+\bm{a}',b+b')$. Similarly, multiplying two such functions gives us:\krzys{i think ill get rid of eisenstein notation coz im not really consistent with it}
\begin{equation}\label{mult}
  \begin{split}
  f_{\bm{a},b}(\bm{x}) \cdot f_{\bm{a}',b'}(\bm{x}) & = ( b - \bm{a}_i \bm{x}^i) \cdot ( b' - \bm{a}'_i \bm{x}^i ) \\
						    & = h_0 + h_i \cdot \bm{x}^i + h_{i,j} \cdot \bm{x}^i \bm{x}^j,
\end{split}
\end{equation}
which yields us a second degree polynomial with coefficients $h_{i,j}$ that can be computed by expanding the parenthesis of the upper equation. The decryption is as before, that is evaluating at $\bm{s}$ and reducing modulo 2. Hence, the scheme is trully homomorphic. Unfortunately, nothing in life comes for free and indeed, the multiplication which took a ciphertexts of size $n+1$, expanded it to the one of size approximately $n^2/2$. As one might imagine this is completely unacceptable from an efficiency point of view and surely not enough for bootstrapping. This is where the main contribution of \cite{fhe-lwe} comes in -- the \textit{re-linearization} technique.
\paragraph{Re-linearization}
The goal is to reduce the ciphertext blow-up for multiplication. As turns out we can actually reduce the result back to just a $n+1$ size assuming something that resembles KDM security or ``circular security'' - i.e. we need to assume that the encryption of a secret key does not leak any information about it. However, the key difference is that we encrypt all of the linear and quadratic terms of $\bm{s}$ but using a different key, call it $\bm{t}$. More precisely, we encrypt numbers $\bm{s}^i$ as well as $\bm{s}^{i,j}$ using the new secret key $\bm{t}$. The adjusted equation \ref{mult} with $\bm{s}$ plugged in for $\bm{x}$ now (approximately) looks like: 
\begin{align*}
  %b^i = \langle \bm{a}^i, \bm{t} \rangle + 2e^i +\bm{s}^i \approx \langle \bm{a}^i, \bm{t} \rangle + \bm{s}^i \\
  %b^{i,j} = \langle \bm{a}^{i,j}, \bm{t} \rangle + 2e^{i,j} +\bm{s}^{i,j} \approx \langle \bm{a}^{i,j}, \bm{t} \rangle + \bm{s}^{i,j} \\
  & h_0 + h_i \cdot \bm{s}^i + h_{i,j} \cdot \bm{s}^i \bm{s}^j\\
  = \; \; & h_0 + h_i \cdot (b^i - \langle \bm{a}^i, \bm{t} \rangle) + h_{i,j} \cdot (b^{i,j} - \langle \bm{a}^{i,j}, \bm{t} \rangle), 
\end{align*}
which is just a linear function in $\bm{t}$! The key take-away is that multiplying the two linear functions and later re-linearizing, gives us another linear function that, when evaluated in $\bm{t}$, outputs the product of the original messages.

\subsubsection{FHE from the ring-LWE}
A step forward in losing the need for KDM security (or circular security) was presented by Z. Brakerski and V. Vaikuntanathan in \cite{fhe_rlwe}. There, they have used the ring-LWE assumptions to create a FHE scheme with provable security for KDM. The scheme is relatively simple thanks to the assumptions that guarantee security based on ring-LWE assumptions. We will begin the construction with the SH scheme and later prove the circular security.

\subsubsection*{Ring-LWE SHE scheme}
The assumptions for this scheme are almost identical as for the ring-LWE, i.e. we take $a$ randomly from some ring that we call $R_q$ and the error term $e$ from some distribution $\chi$ (again, usually taken to be Gaussian). Since we will be basing our scheme on the decision version, we should take $q$ to be prime just like in the previous scheme. For compactness we avoid a precise formulation of the parameters, we refer the reader to the Section \ref{ring-lwe} on the hardness of ring-LWE. The only difference in assumptions is that we will take the secret $s$ also from the same distribution $\chi$ which will be useful for the circular security. Note that this choice does not affect the results presented for ring-LWE. 

Let us first introduce a simpler (symmetric) scheme that is only additively homomorphic. This is not difficult as shown couple of times above. We first take $a$ uniform and sample $s,e \leftarrow \chi$ both from the same error distribution $\chi$. Now, to encrypt a message $m \in R_2 = \Z[x]/(x^n +1)$ that lives in the set of polynomials with binary coefficients, we set 
\[ \bm{c} = (c_0, c_1) = (as +2e +m, -a). \]
To decrypt the message $\bm{c} = (c_0, c_1)$ we compute
\[(c_0 +c_1s) \mod 2 = as+2e+m -as \mod 2 = m. \]
Let us now check the claimed additive homomorphism of the scheme. Assuming $\bm{c'}, \bm{c}$ encrypt messages $m, m'$ respectively using the same secret key, compute
\begin{align*}
	\bm{c'+c} = & (c_0 +c_0', c_1+c_1') = (as +2e +m + a's +2e' +m', -a - a') = \\ & ((a + a')s +2(e+e') +(m+m'), -a - a')
\end{align*}
and see that it decrypts correctly as long as the error is not too big.

It should not come as a surprise that it is the multiplication we should pay most attention to. Let us then compute what a product of the first term $c_0 \cdot c_0'$ will look like
\begin{align*}
	c_0 \cdot c_0' = & (as +2e+m)(a's+2e'+m')\\
	= & \; \; aa's^2 + 2(2ee'+em'+e'm) +mm' \\
	  & + s(a'(2e+m) + a(2e'+m')) \\
	= & \; \; aa's^2 + 2(2ee'+em'+e'm) +mm' \\
	  & + s(a'c_0+ac_0' - 2aa's) \\
	= & \; \; -aa's^2 + s(a'c_0 + ac_0') + 2(2ee'+em'+e'm) +mm'.
\end{align*}
This almost looks like a valid ciphertext for $c_0 \cdot c_0'$ if not for the $-aa's^2$ term. We can fix that by adding one more value to our ciphertext to make it look like $\bm{c}_{mult} = (c_{mult, 0}, c_{mult, 1}, c_{mult, 2})$ where $c_{mult, 2} = c_1c_1'$, $c_{mult, 1} = c_0c_1' + c_0'c_1$ and $c_{mult, 0} = c_0c_0'$. To now decrypt the 3 element ciphertext, compute $c_0 + c_1s +c_2s^2 \mod 2$. It is important to note that we can still add and multiply the ciphertexts of all lengths -- the addition yields the ciphertext of the bigger size of the operands and multiplication yields the ciphertext of size equal to the sum of operands minus one. 

\paragraph{Parameters}
We will skip most of the parameter choices as they often depend on the implementation. Most of them are just like in the standard case presented in the Section \ref{ring-lwe}, i.e. $q$ is prime and $R_q$ is the ring of integers of some cyclotomic field of degree $2^k$. Additionally, instead of taking our messages to be only binary, we shall widen our scope to any element of $R_t$ where $t \in \Z_q^*$ is also prime -- this will be also useful for the KDM security. If we assume that the greatest ``depth'' we can evaluate our polynomial functions at is $D$, we define the general scheme as follows.

\begin{table}
\begin{mdframed}
	$\alg{KeyGen}$\\
	Sample a ring element $s \leftarrow \chi$ and set the secret key vector as $(1, s, s^2, \ldots, s^D)~\in~R_q^{D+1}$. \\
	$\alg{Encrypt}$\\
	Recall that the plaintext space is $R_t$. To encrypt, sample $(a, as +te) \in R_q^2$ where $a$ was uniform and $e \leftarrow \chi$ taken from some distribution $\chi$. Output the $\bm{c} = (c_0, c_1) \in R_q^2$ computed as
	\[c_0 = b +m \in R_q; \; \; \; c_1 = -a.  \]
	While the encryption of a single element lives in $R_q^2$, homomorphic operations might add more elements as seen above. Thus, a more generic form for the ciphertext is $\bm{c} = (c_0, c_1, \ldots, c_D)$ and we can represent any ciphertext like that because padding with zeros does not influence the ciphertext.
	\\
	$\alg{Decrypt}$\\
	Since the general form of the ciphertext is $\bm{c} = (c_0, c_1, \ldots, c_D) \in R_q^{D+1}$, the decryption is a simple $\langle \bm{c, s} \rangle \mod t = \sum_{i = 0}^D c_i s^i \mod t \in R_q$ \\
	$\alg{Evaluate}$\\
	The sum of two ciphertexts $\bm{c, c'}$ is simply the coordinate-wise addition (note that we assume that they are the same length here, for example by padding). For the multiplication, denote $\bm{c} = (c_0, \ldots, c_d)$ and $\bm{c'} = (c_0', \ldots, c_{d'})$ and use $x$ as a \textit{symbolic} variable to compute
	\[ \bigg( \sum_{i = 0}^d c_ix^i \bigg) \cdot \bigg(\sum_{i = 0}^{d'} c_i'x^i \bigg) =
	\bigg( \sum_{i = 0}^{d+d'} \hat{c_i}x^i \bigg) \]
	where the $\hat{c_i}$'s are the result of multiplication modulo $R_q$. The output ciphertext is $\bm{c}_{mult} = (\hat{c_0}, \ldots, \hat{c}_{d+d'})$.

\end{mdframed}
\end{table}

\paragraph{KDM security}
Let us now briefly motivate how this scheme is circularly secure for a linear function. For a more rigorous presentation, see \cite{fhe_rlwe}.

Consider the ciphertext $\bm{c}= (as +2e +s, -a)$ which ``looks like'' and encryption of the secret key $s$. If we now define $a' = a +1$ we have $\bm{c} = (a's +2e, -a'+1)$ we notice that the part $a's+2e$ is exactly a RLWE sample which (by previous section) is completely indistinguishable from an uniform sample -- call it $u$. Therefore, $\bm{c} \approx (u, -a'+1)$ which is a completely uniform pair and so, we cannot tell what the $s$ was.

\subsubsection{Other works}

