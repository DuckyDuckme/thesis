Gentry's work was a true breakthrough. It not only presented the first, fully homomorphic encryption scheme, but also gave researchers a very powerful tool, the \textit{bootstrapping}. From now on, all we need to construct another FHE scheme, is some suitable (one requirement would be to use a scheme based on ring rather than a group) SHE method, apply appropriate "squashing" to obtain the bootstrapping and we are done. In the following years this is exactly what happened in academia and the industry.

This section will mostly serve as a survey of the main developments towards more efficient fully homomorphic encryption using (ideal) lattices and their security based on computational hardness of the underlying problems. We adopt chronological narrative of the sections, starting with the oldest, the GGH scheme from 1997, progressing through works on (ring-)LWE and eventually arriving at the work of Gentry \cite{gentry_phd} on ideal lattices and further developments on FHE. For a good survey on the lattice based cryptography, see for example \cite{two_faces}, \cite{book} chapter 6 or \cite{lattice-survey}.

\subsection{The GGH public key cryptosystem}
We will start this section with a somewhat simpler cryptosystem that was developed by Goldreich, Goldwasser and Halevi and presented in 1997 \cite{ggh}, called the GGH cryptosystem. This scheme, rather than using ideal lattices (i.e. lattices that are also ideals in the ring of integers), relies on general properties of lattices. Namely, the hardness of the SVP and CVP (see section \ref{hardness}).

\subsubsection*{Idea behind the scheme}
The basic GGH cryptosystem, as mentioned before, is based on the problem of finding the closest vector in the lattice $\mathcal{L}$ to a given point in the ambient space $\R^n$. We are given two bases, call them $\Bg$ and $\Bb$. The $\Bb$ will be our public key and $\Bg$ the secret key. The $\Bb$ consists of long and highly non-orthogonal vectors, as opposed to $\Bg$. Our secret message $\bm{m}$ is represented as a binary vector which we will use to form a linear combination $\bm{s} = \sum m_i \bm{v}_i^{bad} \in \mathcal{L}$ of the vectors in $\Bb$. We now add some small and random error $\bm{e} \in \R^n$ to obtain the ciphertext $\bm{c} = \bm{s} + \bm{e} = \sum m_i \bm{v}_i^{bad} + \bm{e} \in \R^n$ - some point that is not in the lattice, but rather, very close to a point in it.\\

To decrypt, we can use our good basis $\Bg$ to represent $\bm{c}$ and, for example Babai's algorithm\footnote{Simply stated, if the vectors of the basis are sufficiently orthogonal to one another, then this algorithm solves \prob{approxCVP}. However, if the Hadamard ratio is too small, the algorithm fails to find the closest vector - \cite{book}.} to find $\bm{v}$ and represent it in terms of the basis $\Bg$ to recover $\bm{m}$. On the other hand, any eavesdropping adversary that is trying to learn our secret, is left with some bad basis that will be of no help in solving the \prob{CVP}.

\subsubsection*{GGH construction - concretely}
%\noindent\fbox{%
%    \parbox{\textwidth}{%
The encryption algorithm can be seen in the Table \ref{ggh-enc}.
\begin{table}[ht]
	\centering
	\begin{mdframed}
$\alg{KeyGen}$:
\begin{itemize}
    \item Pick a basis $(\bm{v}_1, \bm{v}_2, \dots, \bm{v}_n) \subset \Z^n$ such that they are reasonably orthogonal to one another - i.e. with small Hadamard ratio. We will associatie the vectors $\bm{v}_1, \bm{v}_2, \dots, \bm{v}_n$ as the $n$-by-$n$ matrix $\bm{V}$ and let $\mathcal{L}$ be the lattice generated by these vectors. This is our good basis $\Bg$ - the \textbf{private key}.
    \item Pick an $n$-by-$n$ matrix $\bm{U}$ with integer coefficients and determinant $\pm 1$ and compute $\bm{W} = \bm{UV}$. The column vectors $\bm{w}_1, \bm{w}_2, \dots, \bm{w}_n$ of $\bm{W}$ are the bad basis $\Bb$ of $\mathcal{L}$ - the \textbf{public key}\footnote{As an alternative, in \cite{hnf}, Micciancio suggested to use the Hermite Normal Form (HNF) of $\Bg$ which essentially provides the worst possible lattice choice for cryptoanalysis, yet making it the most efficient option.}.
\end{itemize}
$\alg{Encrypt}$:\\
To encrypt a message $\bm{m} = (m_1, m_2, \dots, m_n) \in \Z^n$, choose random small vector $\bm{r} \in \R^n$  and compute $\bm{e} = \bm{mV} + \bm{r}$.\\
$\alg{Decrypt}$:\\
Use Babai's algorithm to compute the vector $\bm{v} \in \Ll$ closest to $\bm{e}$. Finally, compute the $\bm{vW^{-1}}$ to recover $\bm{m}$.
\end{mdframed}
\caption{GGH encryption algorithm}
\label{ggh-enc}
\end{table}


The greatest drawback of GGH is that there were no proofs of security presented along the algorithm, only heuristic assumptions. This motivated researchers to look for possible exploits beased on the choice of parameters. Indeed, this scheme turned out to be insecure for most practical choices of the security parameter only 2 years later, in \cite{break1} and broken completely in \cite{break2}. Nonetheless, the ideas presented there have served as a basis for many schemes that are proven to be secure, like for example LWE, and has led to a plethora of applications.
\subsection{Learning With Errors}
Let us now begin with what went wrong in GGH. Namely, first prove the hardness of a problem, then use it to construct a secure and efficient cryptosystem. In this section we introduce \textit{Learning With Errors} (LWE) problem and the cryptosystem introduced by Oded Regev in \cite{regev} (he won the \href{https://eatcs.org/index.php/component/content/article/1-news/2670-2018-godel-prize}{2018 Gödel Prize} for this work). This very important work in the field of lattice based cryptography is, up to the date, one of the most efficient schemes with an actual proof of security. It has served as a foundation for countless subsequent works in the field due to its versatility. It can be used for digital signature schemes (like \cite{dbs}), key exchange schemes (like \cite{kes}) and encryption schemes which will be mentioned at the end of this section. Arguably the most important contribution is that of laying groundwork for its ring equivalent - the ring-LWE \cite{ring-lwe} that we will introduce after.
